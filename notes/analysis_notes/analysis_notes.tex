\documentclass{article}

\usepackage{geometry}  % Set margins
	\geometry{left=2.2cm,
		right=2.2cm,
		top=2.2cm,
		bottom=2cm}
\parskip 0.15cm
\setlength{\parindent}{0cm}

\usepackage{graphics}
\usepackage{graphicx}
\graphicspath{ {img/} }
\usepackage{float}  % Fancy graphics placement [H] [H!] arguments

\usepackage{amsmath}
\usepackage{amssymb}
\usepackage{textcomp}
\newcommand{\textapprox}{\raisebox{0.5ex}{\texttildelow}}

\usepackage{listings}
\usepackage{xcolor}

\definecolor{codegreen}{rgb}{0,0.6,0}
\definecolor{codegray}{rgb}{0.5,0.5,0.5}
\definecolor{codepurple}{rgb}{0.58,0,0.82}
\definecolor{backcolour}{rgb}{0.95,0.95,0.92}

\lstdefinestyle{mystyle}{
    backgroundcolor=\color{backcolour},
    commentstyle=\color{codegreen},
    keywordstyle=\color{magenta},
    numberstyle=\tiny\color{codegray},
    stringstyle=\color{codepurple},
    basicstyle=\ttfamily\footnotesize,
    breakatwhitespace=false,
    breaklines=true,
    captionpos=b,
    keepspaces=true,
    numbers=left,
    numbersep=5pt,
    showspaces=false,
    showstringspaces=false,
    showtabs=false,
    tabsize=2
}
\lstset{style=mystyle}

\usepackage{cite}
\usepackage{natbib}
\bibliographystyle{agsmnourl}  % Use my custom agsm bibliography template which never includes URLs in articles
\usepackage{url}

\renewcommand{\refname}{\vspace{-1cm}}

\title{\vspace{-1cm}Progress in analyses - Regional Biodiversity - Biomass relationship}
\date{}

\begin{document}

\maketitle{}

\section{Hypotheses and research question}

\begin{itemize}
	\item{Higher species richness will result in higher biomass stocks due to niche complementarity.}
	\item{Species composition will have more effect on biomass than species richness \textit{per se} due to selection effects.}
	\item{Increased aridity will result in a strong richness - biomass relationship}
	\item{There will be a positive effect of abundance evenness oon biomass stocks.}
\end{itemize}

\section{Data description}

\begin{itemize}
	\item{Above ground biomass}
		\begin{itemize}
			\item{SEOSAW} 
			\item{DBH, height, species, wood density, \verb|BIOMASS| package}
			\item{\citet{SEOSAW}}
		\end{itemize}
	\item{Species composition of stems}
		\begin{itemize}
			\item{SEOSAW}
			\item{Species richness, species abundance}
		\end{itemize}
	\item{Aridity}
		\begin{itemize}
			\item{Global Aridity Index (\verb|Global-Aridity_ET0|)}
			\item{Calculated from Worldclim data}
			\item{\citet{Trabucco2009}}
		\end{itemize}
	\item{Fire frequency}
		\begin{itemize}
			\item{MODIS fire return interval}
			\item{\citet{}}
		\end{itemize}
	\item{Stand structural complexity}
		\begin{itemize}
			\item{SEOSAW}
			\item{DBH coef. var.}
			\item{Height coef. var.}
		\end{itemize}
\end{itemize}

\section{Method}

\subsection{Data cleaning}

\begin{enumerate}
	\item{Aggregate Zambia plots}
	\item{Remove stems \textless 5 cm DBH, \textless 1 m height}
	\item{Remove plots with \textless 10 stems ha\textsuperscript{-1}}
	\item{Remove plots with a small area}
	\item{Remove plots that have:}
		\begin{itemize}
			\item{Wood harvesting}
			\item{Farming}
			\item{Livestock grazing}
			\item{Experimental treatments}
		\end{itemize}
	\item{Remove plots which are composition outliers}
\end{enumerate}

\subsection{Data analysis}

Data was log or cube tranformed where appropriate to achieve a normal distribution. All data was standardized to allow direct comparison of effect sizes within path analyses of the Structural Equation Models (SEMs).

Linear models of each bivariate relationship present in the full SEM were analysed to specify the direction of paths. 

A full SEM was specified, after which variables were incrementally removed in order to assess which variables were redundant in explaining variation in Biomass

\end{document}

% Trabucco, A., and Zomer, R.J. 2009. Global Aridity Index (Global-Aridity) and Global Potential Evapo-Transpiration (Global-PET) Geospatial Database. CGIAR Consortium for Spatial Information. Published online, available from the CGIAR-CSI GeoPortal at: http://www.csi.cgiar.org/
