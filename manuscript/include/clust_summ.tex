
% Table created by stargazer v.5.2.2 by Marek Hlavac, Harvard University. E-mail: hlavac at fas.harvard.edu
% Date and time: Mon, Dec 16, 2019 - 11:33:55
\begin{table}[!htbp] \centering 
  \caption{Description of the biogeographical clusters (C1-C5) to which each plot in the study was assigned. Indicator species were generated using Dufrene-Legendre indicator species analysis \citep{Dufrene1997} implemented with \texttt{indval()} from the \texttt{labdsv} R package \citep{labdsv} and represent species which define the given cluster. Dominant species were identified by choosing the species with the largest AGB contribution within each cluster. Numeric values of species richness, stems ha\textsuperscript{-1} and AGB are medians and interquartile ranges (75th percentile - 25th percentile).} 
  \label{clust_summ} 
\begin{tabular}{@{\extracolsep{0pt}} ccccccc} 
\\[-1.8ex]\hline 
\hline \\[-1.8ex] 
{Cluster} & {Dominant species} & {Indicator species} & {N plots} & {Species rich.} & {Stems ha\textsuperscript{-1}} & {AGB (t ha\textsuperscript{-1})} \\
\hline \\[-1.8ex] 
1 & \begin{tabular}[c]{@{}c@{}c@{}} \textit{Julbernadia }spp. \\ \textit{Brachystegia spiciformis} \\ \textit{Baikeaea plurijuga} \end{tabular} & \begin{tabular}[c]{@{}c@{}c@{}} \textit{Diplorhynchus condylocarpon} \\ \textit{Burkea africana} \\ \textit{Pseudolachnostylis maprouneifolia} \end{tabular} & 687 & 11(11.2) & 170(145.3) & 37(34.68) \\ 
\hline
2 & \begin{tabular}[c]{@{}c@{}c@{}} \textit{Julbernadia }spp. \\ \textit{Brachystegia }spp. \\ \textit{Isoberlinia angolensis} \end{tabular} & \begin{tabular}[c]{@{}c@{}c@{}} \textit{Julbernardia paniculata} \\ \textit{Isoberlinia angolensis} \\ \textit{Brachystegia longifolia} \end{tabular} & 757 & 18(17.6) & 215(171.7) & 48.8(43.7) \\ 
\hline
3 & \begin{tabular}[c]{@{}c@{}c@{}} \textit{Spirostachys africana} \\ \textit{Senegalia }spp. \\ \textit{Euclea racemosa} \end{tabular} & \begin{tabular}[c]{@{}c@{}c@{}} \textit{Baikiaea plurijuga} \\ \textit{Senegalia ataxacantha} \\ \textit{Combretum collinum} \end{tabular} & 226 & 10(10) & 165(157.5) & 46(47.81) \\ 
\hline
4 & \begin{tabular}[c]{@{}c@{}c@{}} \textit{Colophospermum mopane} \end{tabular} & \begin{tabular}[c]{@{}c@{}c@{}} \textit{Colophospermum mopane} \\ \textit{Combretum }spp. \end{tabular} & 99 & 7(8.2) & 190(155.7) & 41.5(36.93) \\ 
\hline
\hline \\[-1.8ex] 
\end{tabular} 
\end{table} 
