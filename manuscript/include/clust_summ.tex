
% Table created by stargazer v.5.2.2 by Marek Hlavac, Harvard University. E-mail: hlavac at fas.harvard.edu
% Date and time: Mon, Nov 18, 2019 - 10:53:21
\begin{table}[!htbp] \centering 
  \caption{Description of the biogeographical clusters (C1-C5) to which each plot in the study was assigned. Indicator species were generated using Dufrene-Legendre indicator species analysis \citep{Dufrene1997} implemented with \texttt{indval()} from the \texttt{labdsv} R package \citep{labdsv}. Numeric values are medians and interquartile ranges.} 
  \label{clust_summ} 
\begin{tabular}{@{\extracolsep{0pt}} clccc} 
\\[-1.8ex]\hline 
\hline \\[-1.8ex] 
{Cluster} & {Indicator species} & {Species richness} & {Stems ha\textsuperscript{-1}} & {AGB (t ha\textsuperscript{-1})} \\
\hline \\[-1.8ex] 
C1 & \begin{tabular}[l]{@{}l@{}l@{}} \textit{Diplorhynchus condylocarpon} \\ \textit{Combretum }spp. \\ \textit{Pseudolachnostylis maprouneifolia} \end{tabular} & 7(5) & 316(228.6) & 37.1(27.66) \\ 
\hline
C2 & \begin{tabular}[l]{@{}l@{}l@{}} \textit{Julbernardia paniculata} \\ \textit{Isoberlinia angolensis} \\ \textit{Albizia antunesiana} \end{tabular} & 8(5) & 291(191.7) & 43.8(40.28) \\ 
\hline
C3 & \begin{tabular}[l]{@{}l@{}l@{}} \textit{Burkea africana} \\ \textit{Pterocarpus angolensis} \\ \textit{Baikiaea plurijuga} \end{tabular} & 5(2) & 362(259.4) & 37(42.59) \\ 
\hline
C4 & \begin{tabular}[l]{@{}l@{}l@{}} \textit{Baikiaea plurijuga} \\ \textit{Terminalia randii} \\ \textit{Albizia amara} \end{tabular} & 11(9) & 495(516.8) & 51.7(50.74) \\ 
\hline
C5 & \begin{tabular}[l]{@{}l@{}l@{}} \textit{Colophospermum mopane} \\ \textit{Pseudolachnostylis maprouneifolia} \end{tabular} & 5(5) & 424(226.8) & 50.9(32.7) \\ 
\hline \\[-1.8ex] 
\end{tabular} 
\end{table} 
