\documentclass[12pt,english,a4paper]{report}

\usepackage{natbib}
\usepackage{amsmath}

\setlength{\textheight}{24cm}
\setlength{\headsep}{0cm}

\usepackage{textcomp}
\newcommand{\textapprox}{\raisebox{0.5ex}{\texttildelow}}

\newcommand{\dir}{chapter}

\begin{document}
\title{A regional scale assessment of the biodiversity - ecosystem function relationship in southern African woodlands}
\author{John L. Godlee}
%\date{
%  \today\\
%  \vspace{2cm}
%  {\centering\footnotesize\begin{tabular}[c]{c}
%      \epsfig{file=../preface/2Line2ColCMYK_CS3.eps,height=3cm}\\
%      School of GeoSciences\\
%      Grant Institute\\
%      West Mains Road\\
%      Edinburgh, EH9 3JW\\
%      Scotland
%    \end{tabular}}
%  }
\maketitle
\tableofcontents
%\clearpage
%\listoffigures
%\listoftables
%\listofalgorithms
%\newpage
%\bibliography{}

\section{Introduction}

A global observed decline in biodiversity may lead to a decrease in ecosystem functioning and health, with knock-on effects for the benefits humans derive from ecosystems as ecosystem services \citep{}. Carbon storage as biomass within ecosystems is a commonly studied ecosystem function, as is Gross Primary Productivity (GPP). The storage of atmospheric CO\textsubscript{2} as biomass buffers increases in atmospheric CO\textsubscript{2} caused by human actions. Concentration of biomass in vegetation and soils also indirectly benefits other ecosystem functions, being correlated with ecosystem stability and resilience to disturbance \citep{}. Understanding how ecosystem structure and diversity affects biomass storage is important in order to predict how changes to communities may impact the global carbon balance, with potential impacts for future climate change.

Savannas and sparse woodlands are the dominant vegetation type across southern Africa, spanning \textgreater{}4 million km\textsuperscript{2} across the region. Climatic conditions and biogeography vary across the region, resulting in a diverse range of woodland tree species compositions, which retain the common features of a non-closed canopy and an understorey generalyl dominated by C4 grass species. Southern African woodlands are highly diverse, thought to harbour \textapprox{}8500 plant species of which there are \textgreater{}300 tree species \citep{Frost1996}.

Unlike temperate and wet tropical forests, where ecosystem assemblage is often determined by competition between tree species in an equilibrium state, southern African woodlands are highly structured by disturbance in the form of fire and herbivory, which limit the growth of trees which would otherwise form a closed canopy forest \citep{}. At the dry end of the woodland gradient, woodlands are limited by low precipitation, which allows C4 grasses to compete with tree seedlings.

Other forces acting upon an ecosystem to influence its functionality include climatic factors and the frequency and intensity of disturbances.

Precipitation acts as a key determinant of woodland structure

In southern African woodlands, disturbance as a result of seasonal fires .... Precipitation interacts with disturbance to increase the prevalence of fires by adding grass fuel load and increasing the probability of a fire occurring, and the intensity of the fire when a fire occurs.

Grimes, mass ratio, selection effects and complementarity effects 

In this study, we made the first known attempt to make a regional estimation of the Biodiversity Ecosystem Function Relationship in southern African woodlands. We used aboveground woody biomass and estimates of ecosystem productivity as measures of ecosystem functionality to understand the co-relationship between tree species biodiversity and ecosystem functionality. We compared the relative effects of tree species biodiversity with that of other environmental factors known to affect ecosystem productivity and biomass accumulation such as precipitation and temperature. In acknowledgement of the wide variation in biogeographically determined community composition across the region, we also used previously determined biogeographical clusters \citep{} to understand how species composition as well as species biodiversity \textit{per se} affected ecosystem functionality. Initially, we made four hypotheses: 1) across the region, after other environmental factors had been accounted for, woodland plots with a higher species richness would have a higher aboveground biomass, 2) increased aridity would increase the strength of the biomass tree species richness relationship, 3) species composition would have a greater affect on aboveground biomass than species diversity \textit{per se}. We used Structural Equation Modelling as a preferred method to simultaneously account for environmental factors and biogeographic factors, which often interact in their effect on ecosystem structure and therefore biomass. 


\subsection{Hypotheses}

We hypothesised that plots with a higher species richness will maintain higher biomass stocks.

Species composition will influence above-ground productivity more than species richness \textit{per se}. 

Plots in more arid regions will feature a stronger positive effect of tree species richness on above ground biomass stocks, due to abiotic facilitation effects, despite lower overall tree species richness.

Structural characteristics of the woodland will interact with species composition and richness to provide an indirect path of influecne between species composition and biomass stocks.


\section{Methods}

All analyses were performed in R version 3.6.0

Plot data were collated from a network of plots.

Plots exist along a temperature and precipitation gradient (Figure ).

In each of these plots, each living tree stem \textgreater{}5 cm Diameter at Breast Height (DBH) was measured. For each tree, species, DBH and height to the highest branch material were recorded. 

Plots varied in size across the region. 

Rarefied species richness accounted for variation in plot size and therefore sampling effort across the region.

To ensure that all plots represented true woodland, plots with fewer than 20 stems per Ha were discarded.

Structural Equation Models were compared using different combinations of variables. Path analysis investigated the 

Species richness and the Shannon index (Shannon \& Weaver 1949) (Equation 1) were calculated to assess species diversity.

\begin{equation}
	H = -\sum{}P_i \ln{P_i} / \ln{S} 
\end{equation}


\section{Results}

\section{Discussion}

\end{document}

