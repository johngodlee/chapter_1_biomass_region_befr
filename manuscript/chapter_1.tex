\documentclass[11pt,a4paper]{article}

\usepackage{mynotes}

\title{A regional scale assessment of the biodiversity - ecosystem function relationship in southern African woodlands}
\author{John L. Godlee}
\date{}

\begin{document}

\maketitle
\tableofcontents

\section{Introduction}

A globally observed decline in biodiversity may lead to a decrease in ecosystem functioning and health, with knock-on effects for the benefits humans derive from ecosystems as ecosystem services \citep{}. Carbon storage as biomass within ecosystems is a commonly studied ecosystem function, as is Gross Primary Productivity (GPP). The storage of atmospheric CO\textsubscript{2} as biomass buffers increases in atmospheric CO\textsubscript{2} caused by human actions. Accumulation of biomass in vegetation and soils indirectly benefits other ecosystem functions, being correlated with ecosystem stability and resilience to disturbance \citep{}. Understanding how ecosystem structure and species diversity affects biomass storage is important in order to predict how changes to communities may impact the global carbon balance which will impact future climate change \citep{}.

Savannas and sparse woodlands are the dominant vegetation type across southern Africa, spanning \textgreater{}4 million km\textsuperscript{2} across the region. Climatic conditions and biogeography vary across the region, resulting in a diverse range of woodland tree species assemblages, which retain the common features of an open tree canopy and an understorey generally dominated by C4 grass species \citep{Frost1996}. Southern African woodlands are highly diverse, thought to harbour \textapprox{}8500 plant species of which there are \textgreater{}300 tree species \citep{Frost1996}. Southern African Woodlands have been identified by previous studies as a priority for conservation efforts \citep{Brooks2006, Mittermeier2003, Frost1996}. Many conservation projects in the region currently aim to conserve biodiversity and carbon stocks simultaneously under the banner of REDD+ \citep{Hinsley2015}. A small number of academic studies have shown that above ground woody carbon stocks correlate positively with tree species richness \citep{McNicol2017} \todo{others}, but all have been hampered by a restricted plot network. This study is the first regional analysis of the BEFR in southern African woodlands, using plots which straddle multi-dimensional environmental gradients (Figure \ref{}).

The role of environmental factors in shaping the BEFR is poorly understood \citep{Ratcliffe2017}. Climatic variation is known to affect woody biomass production \citep{Michaletz2014}, so it is important to control for these factors and acknowledge their interaction with woody species diversity \textit{and} biomass stocks to effectively model the effects of biodiversity on biomass stocks in a regional analysis. \citet{Sankaran2005} used data from 854 African field sites to show that mean annual precipitation sets the upper limit for woody cover in these ecosystems, while other factors such as herbivory, fire and soil properties contribute to reducing woody cover below this maximum. \citet{Condit2013} found that dry season intensity was the main determinant of tree species distribution in Panamanian tropical forests, which may affect ecosystem level productivity and thus biomass stocks through selection effects, which may promote the growth of a certain highly productive species only under certain environmental conditions. In European forests \citep{Ratcliffe2017} found a general trend towards stronger positive relationships between tree species richness and various ecosystem functions in more arid environments, suggesting variation in the balance between competitive and facilitative effects along the aridity gradient as the driver of this relationship, an example of the Stress Gradient Hypothesis \citep{Dohn2013}. Water availability imposes a physiological limit on growth rate, which interacts with mortality due to stochastic processes such as fire and herbivory to limit maximum potential biomass stocks. Temperature imposes a similar physiological limit on woody growth rate by limiting metabolic process rates. Temperature and water availability together impact the transpiration rate of a tree, with high temperatures and low water availability limiting growth and potentially causing damage or mortality by causing cavitation of vessels within the tree \citep{Rowland2015a, Fensham2009}. In southern African woodlands however, many species are drought adapted and lose their leaves in the dry season, limiting water loss \citep{Solbrig1996}. The effect of extreme climatic conditions may depend on the degree to which the current biota is adapted to it. Furthermore, across many forested ecosystems, water availability, modulated through precipitation and soil type positively correlates with tree species richness \citep{Vila2005}, meaning that extremely arid areas may be limited in their potential ecosystem function. Species composition may therefore have a greater effect on the interaction between environment and biodiversity - ecosystem function relationships, than species richness \textit{per se}.

Unlike temperate and wet tropical forests, where ecosystem assemblage is often determined by competition between tree species in an equilibrium state \citep{}, southern African woodlands are highly structured by disturbance in the form of fire and herbivory, which limit the growth of trees which would otherwise form a closed canopy forest \citep{}. At the dry end of the woodland gradient, woodlands are limited by low precipitation, which allows C4 grasses to compete with tree seedlings.

An underpinning principle of the Biodiversity-Ecosystem Function Relationship (BEFR) is that of niche complementarity. Species are assumed to differ in their functional niche by the fact that they may coexist in a steady-state ecological community, therefore the more species are present the greater the area of the total fundamental niche of the ecosystem is filled, allowing more efficient use of natural resouerces and higher values of productivity and biomass \citep{}. This theory however, which has been supported by many experiments and observational studies in temperate and wet tropical ecosystems, may not hold in savannas, which are structured by disturbance rather than competition.

Precipitation acts as a key determinant of woodland structure

Grimes, mass ratio, selection effects and complementarity effects 

In this study, we made the first known attempt to make a regional estimation of the Biodiversity Ecosystem Function Relationship in southern African woodlands. We used aboveground woody biomass and estimates of ecosystem productivity as measures of ecosystem functionality to understand the co-relationship between tree species biodiversity and ecosystem functionality. We compared the relative effects of tree species biodiversity with that of other environmental factors known to affect ecosystem productivity and biomass accumulation such as precipitation and temperature. In acknowledgement of the wide variation in biogeographically determined community composition across the region, we also used previously determined biogeographical clusters \citep{} to understand how species composition as well as species biodiversity \textit{per se} affected ecosystem functionality. Initially, we made four hypotheses: 1) across the region, after other environmental factors had been accounted for, woodland plots with a higher species richness would have a higher aboveground biomass, 2) increased aridity would increase the strength of the biomass tree species richness relationship, 3) species composition would have a greater affect on aboveground biomass than species diversity \textit{per se}. We used Structural Equation Modelling as a preferred method to simultaneously account for environmental factors and biogeographic factors, which often interact in their effect on ecosystem structure and therefore biomass. 


\subsection{Hypotheses}

We hypothesised that plots with a higher species richness will maintain higher biomass stocks.

Species composition will influence above-ground productivity more than species richness \textit{per se}. 

Plots in more arid regions will feature a stronger positive effect of tree species richness on above ground biomass stocks, due to abiotic facilitation effects, despite lower overall tree species richness.

Structural characteristics of the woodland will interact with species composition and richness to provide an indirect path of influecne between species composition and biomass stocks.


\section{Methods}

\subsection{Study location}

\subsection{Data collection}

All analyses were performed in R version 3.6.0

Plot data were collated from a network of plots across southern Africa, across 12 countries (Figure \ref{plot_map}). Plots are managed for various monitoring and scientific purposes by a number of researchersPlots used in the study were chosen from the full plot network of 5385 plots to satisfy a number of conditions. Plots were \textgreater{}0.1 ha in area \todo{why}. Only stems \textgreater{}5 cm DBH (Diameter at Breast Height, 1.3 m) were included in analyses. To ensure all plots were within woodland rather than ``grass savanna'', plots with a stem density \textless{}20 stems ha\textsuperscript{-1} were excluded. Grass savannas are considered a separate biome according to \citet{Parr2014} \todo{why}. Plots which had been farmed with livestock or for arable in the previous 30 years were excluded as were plots which had been used in experimental treatments such as fire or herbivore exclusion. Plots created by the Zambian Forestry Commission were originally located in clusters of four adjacent 20x50 m plots. Measurements from these plot clusters were aggregated prior to analysis. This left 1084 plots in 10 countries.

The plots used in this study occurred across a wide cimatic gradient (Figure \ref{env_map}). 

All stems \textgreater{}5 cm DBH were measured within each plot. For each tree, species, DBH and height to the highest branch material were recorded. Height was measured through a variety of means \todo{audit of height methods}


\subsection{Data analysis}
Rarefied species richness accounted for variation in plot size and therefore sampling effort across the region. Rarefied species richness was calculated using the \verb|vegan| package in R, using a sample size of 20 stems.

Species richness and the Shannon index (Shannon \& Weaver 1949) (Equation 1) were calculated to assess species diversity.

\begin{equation}
	H = -\sum{}P_i \ln{P_i} / \ln{S} 
\end{equation}

\subsubsection{Structural Equation Modelling}

Structural Equation Models were compared using different combinations of variables. Path analysis investigated the 

\section{Results}

\section{Discussion}

\end{document}

